\documentclass[12pt,letterpaper]{article}
\usepackage{fullpage}
\usepackage[top=2cm, bottom=2.5cm, left=2.25cm, right=2.25cm]{geometry}
\usepackage{amsmath,amsthm,amsfonts,amssymb,amscd}
\usepackage{lastpage}
\usepackage{enumerate}
\usepackage{enumitem}
\usepackage{fancyhdr}
\usepackage{mathrsfs}
\usepackage{xcolor}
\usepackage{graphicx}
\usepackage{listings}
\usepackage{hyperref}
\usepackage{bbm}
\usepackage{float}


\begin{document}

\begin{center}
\section*{Datathon 2019: Tips and Suggestions for Participants}
\end{center}

\hrule

\vspace{6mm}


\begin{enumerate}[label=\textbf{(\alph*)}]

\item \textbf{Missing Data}

\begin{itemize}
  \item Note that some of the topic IDs listed in training.csv are not actually contained in interest\_topics.csv
  \item It is up to you to decide what to do with this missing information
\end{itemize}

\item \textbf{Imbalanced data}

\begin{itemize}
  \item Choose appropriate evaluation metrics! (F1 score, balanced accuracy, precision, recall, etc.)
  \item Undersampling, oversampling, etc.
  \item Imbalanced-Learn package in Python
  \item Use class weights inside your model, to weigh the smaller class more heavily
\end{itemize}

\item \textbf{Sparse data}

\begin{itemize}
  \item Consider dimensionality reduction methods
  \item Also, consider combining topics (for example, combine all topics within the `Arts and Entertainment' category into just one topic)

\end{itemize}


\item \textbf{Dealing with `big data'}


\begin{itemize}
  \item In R: consider using readr::read\_csv() instead of read.csv()
  \item Python: DictVectorizer may come in handy
  \item Use vectorization (rather than for loops) when possible!
\end{itemize}

\item \textbf{Other tips}

\begin{itemize}
  \item Start working on your presentations early! Explaining your results is often just as important as producing them, and it may take longer than you expect
  \item Using visualizations to explain your work can be very helpful

\end{itemize}




\end{enumerate}

\end{document}